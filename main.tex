\documentclass[twocolumn, 10pt,a4j]{jarticle}

% パッケージ
\usepackage[dvipdfmx]{graphicx}
\usepackage{authblk}
\usepackage{amsmath}
\usepackage{amssymb}
\usepackage{amsfonts}
\usepackage{url}
 
% タイトル
\title{Whitesides' Group: Writing Paper (日本語訳)}
\author{George M. Whitesides}
\date{}

\begin{document}
  % タイトル
  \maketitle

  % 本文
  \section{科学論文とは何か?}
    論文とは,仮説・データ・結論を,読者に伝えるために構成した文書である.
    論文は研究の中核をなす.
    論文を生産しない研究というのは,何もしていないことと等しい.
    研究において「興味深いが未発表である」ことは「存在していない」ことと同じだ.

    研究の目的とは,仮説を立てて検証し,そこから結論を導くこと,そしてその結論を他人に伝えていくことだ.
    「データを集めること」は研究の目的ではない.

    論文とは,完結した研究プロジェクトを保存するための単なる記録文書ではなく,
    進行中のプロジェクトを設計するための骨組みでもある.
    論文の目的と形式を明確に理解すれば,それは研究の構成と実行に対して非常に役立つだろう.
    優れた論文のアウトラインは,そのまま優れた研究計画にもなる.
    したがって,研究の進捗中は,論文のアウトライン/研究計画を書き,そして上書きすべきだ.
    ほとんどの研究では,まず研究計画を立て,最後にアウトラインを決めるが,
    「データ収集が完了してから,結果を論文にまとめる」という方法よりも,「理解・分析・要約・仮説の再構成を継続的に行う」方法のほうが大いに効率的である.

  \section{アウトライン}
    \subsection{なぜアウトライン法なのか?}
    論文の執筆,セミナーの準備,研究の計画立案,などあらゆる作業においてアウトラインが中核をなすことを,ここで強調しておきたい,
    特に,論文執筆においては,アウトラインから文章を書く方法が最も効率的である(私であれ貴方であれ、誰にでも当てはまることだろう).
    ここでいうアウトラインとは,論文の構成(掲載すべきデータを含む)を記した設計図のことである.
    実際,アウトラインとは,「本文の概要」ではなく「目的・仮説・結論につながるデータ群を、注意深く整理・表現したもの」と考えるべきだろう.

    アウトラインそれ自体は,ほとんど文章を含まない.
    もし(指導教官と執筆者で)アウトライン(データとその構成)の詳細まで合意できるのならば,補助的なテキストは正しく組み立てることができる.
    もしアウトラインに合意が得られなければ,どんなテキストを加えたところで役に立たない.
    論文執筆の際,多くの時間が文章記述に投入されてしまう.
    しかし,ほとんどの思考は,データの構成および分析作業に使われるべきである.
    論文のテキストを書き始める前に,数回(場合によってはもっと)アウトラインを書き直すほうが,相対的に時間効率が良い.
    全文を何度も書き直すやり方は,時間効率が悪い.

    論文・報告書・提案書(そして勿論,セミナーのスライド)などの全てに対して,私はアウトライン法に基づき、作業を行っている。
    したがって,私はアウトライン法の使い方を学ぶことも勧めておく.

    \subsection{アウトラインをどう組み立てるべきか}
    古典的なアプローチだが,まず白い紙を用意し,どんな順番でも良いので、論文に関係する思いつく限りの重要事項を書き出してみる.
    そして以下のような簡潔明瞭な質問を自分自身に問いかける.
    「なぜこの仕事をしたのか?」
    「この仕事は何を意味するのか?」
    「どういう仮説を検証するつもりなのか?」
    「実際に検証したものは何か?」
    「検証結果は何か?化合物に対する新しい手法を生み出したか?それは何か?」
    「どのような測定を行ったのか?」
    「何の化合物か?それらはどのように同定されたのか?」.
    考えうる数式・図・反応式も,説明しておく.
    主要なアイデアを捕まえることは必要不可欠だ.
    ある仮説を検証するために研究を始めたとしても,
    実際に手に入れたデータをみると,他の仮説を検証した方が良いと思うかもしれない.
    しかし心配はいらない.2つの仮説を書き出し,仮説・目的・データの最適なコンビネーションを選びとればよい.
    完成した論文上での研究目的と、仕事に取り掛かることを正当化するための研究目的は,往々にして異なるものだ.
    良い科学といえども,そのほとんどは日和見主義的・修正主義的なものだ.

    白い紙に,可能な限り関連事項を書き出せたら,ゴチャ混ぜになった紙を,別の紙を使って整理する.
    この際,書き下した全てのアイデアを,大きく以下の3つに分類する.

    \begin{itemize}
      \item[1.] \mbox{序論}\\ 
      なぜこの仕事をしたのか?
      主たる動機付けや仮説は、一体どういったものだったか?
      \item[2.] \mbox{結果と考察}\\
      どういう結果が得られたのか?
      どのように化合物は合成・同定されたのか?
      何を測定したのか?
      \item[3.] \mbox{結論}\\ 
      結局,論文にどんな意味があるのか?
      どの仮説が肯定or否定されるのか?
      何が分かったのか?
      なぜ,論文は差別化されるのか?
    \end{itemize}

    次に,各章ごとに,さらに細かく整理しよう.
    ここでは「データ」を整理することに集中する.
    データをなるべく明瞭かつ簡潔に表現するように,図・表・反応式を仕上げる.
    この作業はとても時間がかかる.
    私の場合は,最もクリアな図(そして最も美しく見える図)を一つ決めるのに,5-10通りをスケッチすることすらある.

    そして最後に,章立てアウトライン・表・略図・数式など、全てを上手く並べよう.

    掲載すべきデータに全て満足でき(もしくはどんな追加データを集めなければならないかが判明し),適切に配置できたら,そのアウトラインを私に見せて欲しい.
    どこのデータが欠けているのか,そこはどういうデータが出ると考えているのか,あなたの仮説が正しいとすればデータをどう解釈するつもりなのか――それを端的に示しておいて欲しい.
    受け取ったアウトラインは,意見を加えて変更を指示した後,返却する.
    アウトラインが納得行くものになるまで,こういったプロセスをおよそ4-5回は繰り返す(しばしば追加実験を伴う).
    納得が行った時のデータは大抵,最終形(に近いもの)になっているはずだ(すなわちアウトライン上の表・図etcが、結局は論文上の表・図etcそのものになる).

    それが終わったら散文体を心がけ,テキストを書き始めて欲しい.

    プロジェクトの開始初期から,アウトラインとプロポーザルを私とやりとりし始めることが,時間を効率よく使うコツだ.
    どんな状況であれ,「データ収集が完了するまで,アウトラインを書き始めない」ことはNGだ.
    プロジェクトと名のつく全てのものは,「完了」など永遠にやってこない.
    プロジェクトの基本構造を把握でき次第すぐにでも,考えうる論文・アウトラインを提出する.
    そうすることで労力・時間は大幅に節約できる.
    たとえ論文をまとめるまでに,多くの追加実験をすることになったとしても,
    アウトラインを書くひと手間は,研究のガイドとして役立つはずだ.

    \subsection{典型的なアウトライン}
    アウトラインに含めるべきことは次の通り.

    \begin{itemize}
      \item[1.] \mbox{タイトル}  
      \item[2.] \mbox{著者名} 
      \item[3.] \mbox{論文要旨} \\
      書かない.本文の執筆が完了した時に書く.
      
      \item[4.] \mbox{序論} \\
      1-2段落目までを完璧に書き上げる.
      最初の一文には特に注意を払う.
      研究の目的を簡潔に述べ、なぜそれが重要なのかを示すことができれば理想的.
      一般的に、序論は以下のような要素から成る.

        \begin{itemize}
          \item 研究目的
          \item 研究目的の正当性
          \item 研究背景
          \item 読者への手引き
          \item 要約/結論
        \end{itemize}

      \item[5.] \mbox{結果と考察} \\
      結果と考察は得てして不可分である.
      この項目は主軸となるトピックに従い,構成されねばならない.
      明確な構成にすべく,各項目ごとに太字の小見出しをつける.
      こうすることで,読者は自分の知りたい箇所を,ざっと見て見つけやすくなる.
      項目見出しとして適切なフレーズの例を,以下に示す.

        \begin{itemize}
          \item アルカンチオールの合成
          \item ビシナルジオール単位の絶対配置
          \item ヒステリシスは表面の粗さと相関する
          \item 速度定数の温度依存性
          \item 溶媒の極性に従って自己交換速度は減少する
        \end{itemize}

      \item[6.] \mbox{結論} \\
      アウトラインの中で,結論は「短い語句・文のリスト」としてまとめる.
      特別な強調がいる場合を除いて、「5.結果」で書いたことを繰り返さない.
      要約にならないように,結論だけを書く.
      結論には,新しく,高いレベルの分析を付与するべきであり,
      研究の意義をはっきりと示さなくてはならない.
       
      \item[7.] \mbox{実験} \\
      「5.結論」で示した順番通りに,正しく対応させつつ,
      「7.実験」の全ての段落見出しを含めておく.

    \end{itemize}

    \subsection{まとめ}
    \begin{itemize}
      \item なるべくプロジェクトの初期に,アウトラインを書き始める.「プロジェクトの完了」を待ってから,というのはNG.終わりなど永遠にやってこない.
      \item アウトラインと論文は,簡単に埋め込めるデータ(つまり表・数式・図・反応式)を主体に構成する.テキスト主体はNG.
      \item 研究の時間軸に沿わず,重要度の順でまとめる.論文を書く上で,トピックごとの重みづけを考えることは重要だ.実際にやった順番どおりデータを示す――すなわち,愛すべき初期の失敗例から始め,成功をクライマックスとして持ってくるよう,実験事実を並べる――ことを初心者は往々にしてやりがちだ.このやり方はまったく正しくない.最も重要度の高い結果から書き始めること.次に2番目に重要な結果を記す(もしあれば、だが).大抵の読者は、その偉業にどうやって辿り着いたかという「過程」には興味がない.偉業の「中身」だけを知りたがる.それに長い論文よりは、短い方が読むのも楽だ.
    \end{itemize}

    \subsection{書式に関する注意点}
    \begin{itemize}
      \item 名詞を形容詞として使わない.
      \item 被指示語が明確になるよう,「this」のあとには常に名詞をおく.
      \item 実験結果は過去形で統一する.
      \item 可能な限り,常に能動態を使う.
      \item 全ての比較対象を網羅する.
      \item 全ての論文はダブルスペースで書く(1スペースや1.5スペースはNG).コロン(:)と文末ピリオド(.)の後には2スペース置く.余白は十分にとる.
      \item 全ての論文はアメリカ化学会(ACS)書式で書く.
      \begin{itemize}
        \item[-] 学術雑誌
        \item[-] 研究グループの過去論文
        \item[-] The ACS Handbook for Authors
      \end{itemize}
    \end{itemize}

  \section{参考文献}
  \begin{itemize}
    \item The original work by G. M. Whitesides, 2004 \\ \url{https://gmwgroup.harvard.edu/files/gmwgroup/files/895.pdf}
  \end{itemize}
\end{document}


% https://www.chem-station.com/blog/2009/12/whitesides_group_writing_a_pa.html
% https://gmwgroup.harvard.edu/files/gmwgroup/files/895.pdf
